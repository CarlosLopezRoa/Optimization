\documentclass[a4paper,11pt]{article}
\usepackage{cite}
\usepackage{listings}
\usepackage[english]{babel}
\usepackage[T1]{fontenc}
%\decimalpoint 
\usepackage{amsmath}
\usepackage{amssymb}
\usepackage{graphicx}
\usepackage[]{circuitikz}
\setlength{\oddsidemargin}{0in}
\setlength{\topmargin}{-.8in}
\setlength{\textheight}{9.7in} \setlength{\textwidth}{6.5in}
\usepackage{color,hyperref}
\definecolor{darkblue}{rgb}{0.0,0.0,0.3}
\hypersetup{colorlinks,breaklinks,
            linkcolor=darkblue,urlcolor=darkblue,
            anchorcolor=darkblue,citecolor=darkblue}
\providecommand*\url[1]{\href{#1}{#1}}
\renewcommand*\url[1]{\href{#1}{\texttt{#1}}}

\newcommand{\bm}[1]{\boldsymbol{#1}}
\newcommand{\bh}[1]{\boldsymbol{\hat{#1}}}
\newcommand{\bt}[1]{\boldsymbol{\tilde{#1}}}
\newcommand{\bbar}[1]{\boldsymbol{\bar{#1}}}
\newcommand{\mbf}[1]{\ensuremath{\mathbf{#1}}}
\newcommand{\ode}[2]{\ensuremath{\frac{\mathrm{d} #1}{\mathrm{d} #2}}}
\newcommand{\odet}[2]{\ensuremath{\tfrac{\mathrm{d} #1}{\mathrm{d} #2}}}
\newcommand{\oden}[3]{\ensuremath{\frac{\mathrm{d}^#3 #1}{\mathrm{d} #2^#3}}}
\newcommand{\pde}[2]{\ensuremath{\frac{\partial #1}{\partial #2}}}
\newcommand{\pdet}[2]{\ensuremath{\tfrac{\partial #1}{\partial #2}}}
\newcommand{\pden}[3]{\ensuremath{\frac{\partial^{#3} #1}{\partial
      #2^{#3}}}}
\newcommand{\sub}[1]{\ensuremath{_{\rm{#1}}}}
\newcommand{\arriba}[1]{\ensuremath{^{\rm{#1}}}}
%
\newcommand{\N}{\ensuremath{\mathbb{N}}}
\newcommand{\R}{\ensuremath{\mathbb{R}}}
\newcommand{\C}{\ensuremath{\mathbb{C}}}
\newcommand{\ee}[1]{\ensuremath{\mathrm{e}^{#1}}}
\newcommand{\hdos}{\ensuremath{\mathrm{H}_2}}
\newcommand{\COdos}{\ensuremath{\mathrm{CO}_2}}
\newcommand{\ATP}{\ensuremath{\mathrm{ATP}}}

\newcommand{\dt}{\ensuremath{\mathrm{d}t}}
\newcommand{\dtau}{\ensuremath{\mathrm{d}\tau}}
\newcommand{\DV}{\ensuremath{\Delta V}}

\DeclareMathOperator{\Li}{\mathcal {L}^{-1}}
\DeclareMathOperator{\Lin}{\mathcal {L}^{-1}}
\DeclareMathOperator{\sinc}{\text{sinc}}
\DeclareMathOperator{\sign}{\mathrm{sign}}

\newtheorem{deff}{Definition}
\usepackage[utf8x]{inputenc}
%%

\title{Homework 1\\ Optimization}
\author{Carlos López Roa}
\date{\today}
\pdfinfo{%
  /Title    ()
  /Author   (CLR)
  /Creator  ()
  /Producer ()
  /Subject  ()
  /Keywords ()
}
\begin{document}
\maketitle
%%% Content
Show that the following functions are log-concave\footnote{From  \emph{Boyd, S., \& Vandenberghe, L. (2002). Convex Optimization} problem 3.49}, i.e.
\begin{deff}
$f$ is log-concave if 
\begin{equation}
\begin{aligned}
f(\theta x + (1-\theta y)) \geq f(x)^\theta f(y)^{1-\theta}
\end{aligned}
\end{equation}
 or equivalently 
\begin{equation}
\begin{aligned}
\log f(\theta x + (1-\theta)y) \geq \theta \log f(x) + (1-\theta)\log f(y)
\end{aligned}
\end{equation}
$$\forall x,y  \in dom f; 0 \leq \theta \leq 1$$ 
\end{deff}

\begin{enumerate}
\item Logistic function
\begin{equation}
\begin{aligned}
\frac{e^x}{(1+e^x)} && \text{dom} f = \mathbb{R}
\end{aligned}
\end{equation}

\begin{equation}
\begin{aligned}
f(\theta x + (1-\theta) y ) = \frac{e^{\theta x + (1-\theta) y}}{1+e^{\theta x + (1-\theta)y}} &\geq \frac{e^{\theta x + (1-\theta) y}}{(1+e^x)^\theta(1+e^y)^{1-\theta}} = f(x)^\theta f(y)^{1-\theta} \\
1+e^{\theta x} e^{(1-\theta)y} &\leq (1+e^x)^\theta(1+e^y)^{1-\theta} \\ 
 \theta \log e^{ x} + (1-\theta) \log e^{y}&\leq \theta \log (1 + e^x) + (1-\theta)\log (1+e^y)
\end{aligned}
\end{equation}

Which holds since $\log (e^x) \leq \log (e^x + c)$ , $\forall \, c \geq 0$  

\hspace{15.5cm}$\square$

\item Harmonic mean : 
\begin{equation}
\begin{aligned}
f(x) &= \frac{1}{\sum_i^nx_i} & \text{dom} f = \mathbb{R}_{++}^n
\end{aligned}
\end{equation}

Recall $\min{x_i}\leq H(x)\leq n\min{x_i}$. With $H(x) = n f(x)$ the conventional Harmonic mean. Then 

\begin{equation}
\begin{aligned}
f(\theta x + (1-\theta y)) &\geq f(x)^\theta f(y)^{1-\theta}\\
n\min\{\ \theta x_i +(1-\theta)y_i\} &\geq \min \{ x_i \}^{\theta} \min \{ y_i \}^{1-\theta}
\end{aligned}
\end{equation}
Let $x_k,y_k$ be the minimums respectively.

\begin{equation}
\begin{aligned}
\theta x_k +(1-\theta)y_k &\geq x^\theta_k y^{1-\theta}_k
\end{aligned}
\end{equation}
Which holds $\forall \, x_k, y_k \in \mathbb{R}^n_{++}$ and $0\leq \theta \leq 1$

\hspace{15.5cm}$\square$

\newpage
\item Product over sum:
\begin{equation}
\begin{aligned}
f(x)&=\frac{\prod^n_{i=1}x_i}{\sum^n_{i=1}x_i} & \text{dom} f = \mathbb{R}_{++}^n
\end{aligned}
\end{equation}

\begin{equation}
\begin{aligned}
\log f(x) &= \log \prod^n_{i=1} x_i - \log \sum^n_{i=1} x_i \\
&= \sum^n_{i=1}\log x_i - \log \sum^n_{i=1}x_i
\end{aligned}
\end{equation}

\begin{equation}
\begin{aligned}
f(\theta x + (1-\theta)y) &\geq f(x)^\theta f(y)^{1-\theta} \\
\frac{\prod^n_{i=1} \theta x + (1-\theta)y}{\sum ^n_{i=1}  \theta x + (1-\theta)y} &\geq \left(\frac{\prod ^n_{i=1} x_i}{\sum ^n_{i=1} x_i}\right)^\theta\left(\frac{\prod ^n_{i=1} y_i}{\sum ^n_{i=1} y_i}\right)^{1-\theta} \\
&\geq \frac{\prod x^\theta_i y_i^{1-\theta}}{(\sum x_i)^\theta (\sum y_i)^{1-\theta}}
\end{aligned}
\end{equation}
Which holds by the previous result

\hspace{15.5cm}$\square$

\item Determinant over trace:
\begin{equation}
\begin{aligned}
f(X)&=\frac{\det X}{\text{tr} \, X} & \text{dom} f = \mathbb{S}_{++}^n
\end{aligned}
\end{equation}

Since $X \in \mathbb{S}_{++}^n$ we can always find $\Lambda$ the associated diagonal matrix with eigenvalues $\lambda_k >0$ and we may write the invariants:

\begin{equation}
\begin{aligned}
f(X) = f(\Lambda) = \frac{\det \Lambda}{\text{tr} \, \Lambda} = \frac{\prod \lambda_i}{\sum \lambda_i}
\end{aligned}
\end{equation}
\begin{equation}
\begin{aligned}
\log f(x) &= \sum \log \lambda_i - \log \sum \lambda_i
\end{aligned}
\end{equation}

Which is the previous result

\hspace{15.5cm}$\square$

%%%
\end{enumerate}
\end{document}